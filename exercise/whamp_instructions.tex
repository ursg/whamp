\documentclass[a4paper,10pt]{article}
\usepackage[utf8]{inputenc}

\usepackage[left=2cm,right=2cm,top=2cm,bottom=2cm]{geometry}

\usepackage{hyperref}

%opening
\title{WHAMP instructions/Advanced space plasma exercise}
\author{Y. Pfau-Kempf}

\begin{document}

\maketitle

% \begin{abstract}

% \end{abstract}

\section{Installing WHAMP}

\begin{enumerate}
   \item Get the WHAMP code by downloading at \url{https://github.com/ykempf/whamp}
   or using the git command \\ \\
   \verb=$ git clone git@github.com:ykempf/whamp.git whamp= \\ \\
   which will download the code into the \verb=whamp= directory.
   
   \item Go to the code folder \\ \\
   \verb=$ cd whamp/src/= \\ \\
   and compile the code \\ \\
   \verb=$ make= \\ \\
   and do not worry about all the warnings. The file called \verb=whamp= is the executable you will need.
\end{enumerate}

By running this \verb=whamp= executable you can use WHAMP on a point-by-point basis as the instructions in the manual tell you to do. But that is a bit painstaking so we use a script to automatise the process.


\section{Running WHAMP from the Python scripts}

\begin{enumerate}
   \item Copy the \verb=whamp= binary to the \verb=exercise= directory. \\ \\
   \verb=$ cp whamp ../exercise=
   
   \item Run the script with Python \\ \\
   \verb=$ cd ../exercise= \\
   \verb=$ python3 <file>.py= \\ \\
   and look at the 3D dispersion surfaces.
   
   \item If you wish to save a png instead of displaying the result on screen, swap the \verb=#= comment character between the last two lines of the
script so that you use \\ \\
   \verb+matplotlib.pyplot.savefig('plot.png', dpi=DPI)+ \\ \\
   instead of \\ \\
   \verb=matplotlib.pyplot.show()= \\ \\
   which will write the file \verb=plot.png= to disk.
\end{enumerate}


\end{document}
